\documentclass[letterpaper,12pt]{article}
\usepackage[utf8]{inputenc}
\usepackage[activeacute,spanish]{babel}
\usepackage{amsmath}
\usepackage{amsfonts}
\usepackage{enumerate}
\usepackage{float}
\usepackage{indentfirst}
\usepackage{graphicx}
\usepackage{url}
\usepackage{multicol}
\usepackage{subfigure}
\usepackage[position=bottom]{subfig}
\usepackage{geometry}
\usepackage{fullpage}
\usepackage{algorithmic}
\usepackage{algorithm}
\usepackage{verbatim}
\usepackage[usenames]{color}
\renewcommand{\algorithmicforall}{\textbf{for each}}
\setlength\parindent{0pt}


\begin{document}



\begin{minipage}[t]{0.6\textwidth}
{\Large \textbf{INF152} Estructuras Discretas}

{\large Profesores: M. Bugueño -- S. Gallardo}

Universidad Técnica Federico Santa María

Departamento de Inform\'atica -- Octubre 15, 2021.

\end{minipage}
\hfill
\begin{minipage}[t]{0.35\textwidth}
%RELLENE CON SUS DATOS PERSONALES:
\textbf{Ayudantes}\\
valentina.arostica@sansano.usm.cl\\
bryan.gonzalezr@usm.cl
\end{minipage}

\vspace{0.8cm}

\begin{center}
    \Huge{Pseudocódigo}
\end{center}


%%%%%%%%%%%%%%%%%% Fin Header %%%%%%%%%%%%%%%%%%%%%%%%

\vspace{0.4cm} 


\section{¿Qué es el pseudocódigo?}
Es un lenguaje creado especialmente para la realización de algoritmos; la característica
principal de éste es que se pensó para el entendimiento del humano y no el de la máquina.
Por ello es que se considera un lenguaje sencillo. Como todos los algoritmos se deben
ejecutar en una máquina, es necesario traducir el pseudocódigo a un lenguaje de
programación, siendo considerado un borrador, por esto es utilizado en textos donde no está
definido un lenguaje de programación en particular, haciendo de él uno universal.\\

El pseudocódigo es considerado un lenguaje de \textbf{alto nivel} (se caracteriza por
ser más entendible por el humano que los de un nivel inferior) y posee una estructura
secuencial.

\section{Características}
El pseudocódigo comparte muchas características con lenguajes de programación formales, entre las que podemos encontrar:
\begin{itemize}
    \item \textbf{Variables: } Son espacios de memoria que pueden cambiar su contenido a lo largo de la ejecución de su pseudocódigo, y se accede a través del identificador (nombre)\\[0.1cm]
    \textbf{Constantes: } No cambia en el transcurso de la ejecución
    \item \textbf{Entradas y salidas(INPUT/OUTPUT): } Las entradas son los datos que el usuario
    ingresa para procesar en el programa. Y las salidas son las respuestas que el
    programa entrega al usuario.
    \item \textbf{Instrucciones de control: } Principalmente se utilizan 2: If y for (o foreach), que vienen en el paquete latex que detallaremos más adelante ($\backslash$usepackage\{algorithm\} y $\backslash$usepackage\{algorithmic\})
\end{itemize}
\vspace{5cm}
\section{Paquete \textit{``$algorithm$"} y \textit{``$algorithmic$"}}
{\LaTeX} es una muy buena herramienta para plasmar pseudocódigo en un pdf, pues existe el paquete $algorithm$, el cual facilita mucho la creacion de pseudocódigo, además de tener un diseño muy estético.
\subsection{Introduccion a alogorithm}
Lo primero que deben hacer para trabajar con el paquete, es incluirlo en su código {\LaTeX}, a través de $\backslash$usepackage\{algorithm\} y $\backslash$usepackage\{algorithmic\}. Ambos paquetes son complementarios\\[0.1cm]
Para comenzar un algoritmo con este paquete, lo primero que debemos hacer es crear el ambiente en el que trabajaremos. Esto se puede ver a mas detalle en el siguiente ejemplo (para ver el código, por favor revise el archivo .tex): 
\begin{algorithm}[H]
  \begin{algorithmic}[1]
        \STATE{\textbf{procedure} $test$}
        \STATE variable $\leftarrow$ 1
        
  \end{algorithmic}
  \caption{Algoritmo de prueba}
  \label{alg:SDSA}
\end{algorithm}
En el ejemplo podemos apreciar como asignar un valor a una variable, algo que es muy útil para resolver problemas en pseudocódigo. Para cada linea de codigo (que no sea alguna instruccion de control), utilizaremos $\backslash$STATE. Antes de definir nuestra variable, por ejemplo.\\[0.1cm]

También habrán notado que comenzamos con un \textbf{procedure}. Para efectos de este curso, siempre debemos comenzar con \textbf{procedure}, pues es a través de esta sintaxis que definimos el nombre del algoritmo que estamos trabajando, y eventualmente podríamos llamarlo dentro de otro.

\subsection{Condicionales}
Para utilizar condicionales en el pseudocódigo latex, el paquete viene con un comando incluido. Si queremos utilizarlo, en vez de $\backslash$STATE, debemos utilizar $\backslash$IF, luego escribir los condicionales, y en las lineas siguientes los argumentos del condicional.\\
Para terminar el condicional, se \textbf{DEBE} colocar un $\backslash$ENDIF\\[0.1cm]
A continuacion, se mostrará un ejemplo:
\begin{algorithm}[H]
  \begin{algorithmic}[1]
        \STATE{\textbf{procedure} $condicional$}
        \STATE variable $\leftarrow$ 1
        \IF{variable == 1}
        \STATE variable2 $\leftarrow$ 2
        \ENDIF
  \end{algorithmic}
  \caption{Algoritmo de prueba condicional}
  \label{alg:testing2}
\end{algorithm}
Adicionalmente, los condicionales también pueden hacer uso de un else if (como el ``elif" de python), y pueden ser anidados, como se puede ver en el siguiente ejemplo:
\begin{algorithm}[H]
\begin{algorithmic}[1]
        \STATE{\textbf{procedure} $condicional compuesto$}
        \STATE variable $\leftarrow$ 1
        \IF{variable == 1}
        \STATE variable2 $\leftarrow$ 2
        \IF{variable2 == 2}
        \STATE variable3 $\leftarrow$ 5
        \ENDIF
        \ELSIF{variable == 3}
        \STATE variable2 $\leftarrow$ 3
        \ENDIF
  \end{algorithmic}
  \caption{Condicional compuesto}
  \label{alg:testing3}
\end{algorithm}
\subsection{Ciclos}
Como ya deben conocer de otros lenguajes de programación, los ciclos principalmente son los for (o foreach en algunos lenguajes), y los while.\\
En {\LaTeX}, utilizando las librerías anteriormente vistas, podemos escribir los ciclos a través de los comandos $\backslash$FOR y $\backslash$WHILE para representar cada tipo de ciclo. Cabe recalcar que al igual que los condicionales, hace falta cerrar el ciclo.\\[0.1cm]
Veamos un ejemplo para cada uno:\\
\begin{minipage}[t]{0.42\textwidth}
\begin{algorithm}[H]
\begin{algorithmic}[1]
        \STATE{\textbf{procedure} $usoFor$}
        \STATE palabra $\leftarrow$ ``Paralelepipedo"
        \FOR{letra \textbf{in} palabra}
        \PRINT{letra}
        \ENDFOR
  \end{algorithmic}
  \caption{Ciclo For}
  \label{alg:testing4}
\end{algorithm}
\end{minipage}
\hspace{0.5cm}
\begin{minipage}[t]{0.42\textwidth}
\begin{algorithm}[H]
\begin{algorithmic}[1]
        \STATE{\textbf{procedure} $usoWhile$}
        \STATE palabra $\leftarrow$ ``Paralelepipedo"
        \STATE contador $\leftarrow$ 0
        \WHILE{contador $\leq$ len(palabra)}
        \PRINT{palabra[contador]}
        \STATE contador $\leftarrow$ contador+1
        \ENDWHILE
  \end{algorithmic}
  \caption{Ciclo While}
  \label{alg:testing5}
\end{algorithm}
\end{minipage}\\[0.2cm]
Ps. Notese que en este caso, los algoritmos quedaron uno al lado del otro, esto es posible gracias al ambiente minipage, los invitamos a investigarlo.
\end{document}
