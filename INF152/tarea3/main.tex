\documentclass[letterpaper,10pt]{article}
\usepackage[utf8]{inputenc}
\usepackage[spanish,mexico]{babel}
\usepackage{amsmath}
\usepackage{amsfonts}
\usepackage{enumerate}
\usepackage{float}
\usepackage{indentfirst}
\usepackage{graphicx}
\usepackage{url}
\usepackage{multicol}
\usepackage{geometry}
\usepackage{fullpage}
\usepackage{hyperref}
\usepackage{tikz}
\usetikzlibrary{shapes}
\usepackage{makecell}
\usepackage{textcomp}
\usepackage{tikz}
\usetikzlibrary{arrows,petri,topaths,shapes,automata}
\usepackage{tkz-berge}

\usepackage[position=bottom]{subfig}

\setlength\parindent{0pt}

\tikzset{
    %Define standard arrow tip
    >=stealth',
    % Define arrow style
    pil/.style={
           ->,
           thick,}
}

\def\checkmark{\tikz\fill[scale=0.4](0,.35) -- (.25,0) -- (1,.7) -- (.25,.15) -- cycle;}

\begin{document}

\thispagestyle{empty}
 	
\begin{minipage}[t]{0.6\textwidth}

{\LARGE \textbf{INF152} Estructuras Discretas}

{\large Profesores: Margarita Bugueño, Sebastián Gallardo.}\\
{\large Ayudantes: Valentina Aróstica, Bryan González, Sofía Mañana y Sofía Riquelme}

Universidad T\'ecnica Federico Santa Mar\'{\i}a

Departamento de Inform\'atica -- CSJ - CC 

Diciembre 03, 2021

\end{minipage}
\hfill
\begin{minipage}[t]{0.3\textwidth}
Nombre:

\begin{tabular}{|c|}\hline
%%%%%%%%%%%%%%%%%%%%%%%% !!!! LEER !!!!  %%%%%%%%%%%%%%%%%%%%%%%%%%%%%%
% Borre los caracteres ~ y los signo peso $, para poner su nombre aquí: %
Diego Eduardo Paz Letelier\\\hline
\end{tabular}

\vspace{0.1cm}

Rol:

\begin{tabular}{|c|c|c|c|c|c|c|c|c|c|c|}\hline
%%%%%%%%%%%%%%%%%%%%%%%% !!!! LEER !!!!  %%%%%%%%%%%%%%%%%%%%%%%%%%%%%%
% NO borre los caracteres ampersand &, entre ellos ponga cada dígito de su rol:%
%EJEMPLO: 2 & 0 & 2 & 0 & 7 & 3 & 5 & 0 & 0 & 0
2 & 0 & 2 & 0 & 0 & 4 & 5 & 0 & 2 & - & k\\\hline
\end{tabular}
\end{minipage}

\vspace{0.3cm}

\begin{center}
    \huge Tarea 3
\end{center}


\section{Reglas generales}
\textbf{Objetivo:}\\
Profundizar y pulir su manejo de \LaTeX, y demostrar su dominio de los contenidos relacionados con el Certamen 3.

\begin{itemize}
    \item Debe investigar por su cuenta la sintáxis de \LaTeX $~$para adquirir las herramientas que le permitan elaborar un desarrollo claro, ordenado, y \textbf{formal} en cada pregunta.
    
    \item Para resolver cada problema, debe hacer uso de los contenidos, algoritmos y métodos aprendidos en el curso. Si su respuesta final es correcta, pero se ha utilizado un método distinto al enseñado en clases no se asignará puntaje. \textbf{Ídem si entrega resultados sin desarrollo.}
    
    \item No se permite adjuntar/incluir fotografías de grafos o tablas en este archivo. Ya sean dibujadas a mano, en word, en herramientas virtuales de dibujo, etc. Si los grafos y tablas no están hechas en \LaTeX no obtendrá puntaje.
    
    \item La tarea debe realizarse de manera individual. Si se detecta copia, se notificará al profesor y se evaluará el trabajo de los involucrados con la nota mínima.
    
    \item Para revisar sus trabajos se usará el editor Visual Studio Code o en su defecto, el editor Overleaf.
    
\end{itemize}

\section{Problemas}
\begin{enumerate}

    \item Considere la siguiente red de tuberías de la empresa \textit{Aguas Sansanas\texttrademark}, descrita a continuación por sus capacidades de flujo de agua [$m^3/s$]:
    
    \begin{center}
        \begin{tabular}{|c|c|c|c|c|c|c|c|c|}
        \hline
        $\rightarrow$& s & 1 & 2 & 3 & 4 & 5 & 6 & t\\
        \hline
        s& - & 8 & 12 &  &  &  &  & \\
        \hline
        1&  & - &  & 7 &  &  &  & \\
        \hline
        2&  & 4 & - &  & 3 &  & 6 & \\
        \hline
        3&  &  &  & - & 5 &  &  & 8\\
        \hline
        4&  &  &  & 4 & - & 9 &  & \\
        \hline
        5&  &  &  &  &  & - &  & 12\\
        \hline
        6&  &  &  &  & 8 & 2 & - & \\

        \hline

    \end{tabular}
    
    \end{center}
    
    \begin{enumerate}
        \item[a)] Determine \textbf{claramente} el flujo máximo adimisible por esta red, \textbf{justificando formalmente} su respuesta, \textbf{mostrando la red de flujo final.} Debe mostrar todo su procedimiento.(puede copiar y pegar, cuantas veces necesite, el código \LaTeX $~$del flujo inicial haciendo los cambios necesarios para mostrar el paso a paso). (40\%)
        \item[b)] Demuestre que el flujo encontrado es máximo.  (30\%)
    \end{enumerate}
    
    \textbf{Solución:}
    \vspace{1cm}
    
    %Escriba su respuesta aquí
    
    
    \item Considerando la siguiente cartilla, encuentre el $|G|$ e identifique cuales de los movimientos son permutaciones conjugadas entre sí. Justifique su respuesta. (30 \%)
    \\
    \textit{Hint:} Con movimiento se refiere a todas las rotaciones o reflexiones que se pueden realizar a la figura para que estas sean parte de  $|G|$.
    
    
\centering
 \begin{tikzpicture}[auto, node distance=0.6 cm,semithick]
\tikzstyle{every state}=[rectangle,draw,minimum size=0.6cm] 
\node[state] (0) {$A$};
\node[state] (7) [right of=0] {$B$};
\node[state] (1) [below of=0] {$D$};
\node[state] (2) [left of=1] {$C$};
\node[state] (3) [right of=1] {$E$};
\node[state] (4) [below of=2] {$F$};
\node[state] (6) [right of=4] {$G$};

\draw (0,0) circle (0.2cm);
\draw (0.6,0) circle (0.2cm);
\draw (0,-0.6) circle (-0.2cm);
\draw (-0.6,-0.6) circle (0.2cm);
\draw (0.6,-0.6) circle (0.2cm);
\draw (-0.6,-1.2) circle (0.2cm);
\draw (0,-1.2) circle (0.2cm);

\end{tikzpicture}\\
\label{cartilla}
    
\end{enumerate}

\textbf{Solución:}
    \vspace{1cm}

Los movimientos que son permutaciones conjugadas son:
\begin{itemize}
    \item Identidad $\rightarrow$ id
    \item Rotacion en 180° $\rightarrow$ (A G)(F B)(C E)(D)
    \item Reflexión en 45° (Diagonal paralela a A E) $\rightarrow$ (A C)(B F)(E G)(D)
    \item Reflexión en 135° (Diagonal B F) $\rightarrow$ (A E)(C G)(D)(B)(F)
\end{itemize}


\section{Descuentos}
\begin{table}[H]
    \centering
    \begin{tabular}{|c|c|c|}
    \hline
        & \textbf{Criterio} & \textbf{Descuento} [ptos] \\
    \hline
        1 & Copia y/o plagio & 100 \\
    \hline
        2 & Cada Warning al compilar su \LaTeX & 5\\
    \hline
        3 & Cada error de compilación en Overleaf & 100 \\
    \hline
        4 & \makecell{Entrega un archivo comprimido en un formato\\ que no sea .zip} & 5 \\
    \hline
        5 & No pone su nombre en el documento \LaTeX & 5\\
    \hline
        6 & No pone el Rol en el documento \LaTeX & 5\\
    \hline
        7 & Entrega tarde (descuento por cada hora) & 10\\
    \hline
        8 & \makecell{Entrega resultados o respuestas pero no muestra el\\ paso a paso, es decir no muestra un desarrollo elaborado} &  100\% (de la pregunta) \\
    \hline
        9 & \makecell{Inserta fotografías o imágenes en lugar de respuestas\\ elaboradas con código \LaTeX} & 100\% (de la pregunta)\\
    \hline
    \end{tabular}
    \caption{Tabla de descuentos}
    \label{tab:desc}
\end{table}

\section{Entrega}
\begin{itemize}
    \item Debe entregar una carpeta comprimida en .zip que contenga el (o los) archivo(s) .tex con sus respuestas, y las imágenes que van incluidas en el proyecto. El nombre del .zip debe ser su nombre y apellido. Ejemplo:\\ \textbf{Nombre\_Apellido.zip}. En este caso no es necesario agregar el PDF, dado que se compilará cada proyecto a la hora de revisar.
    
    \item \textbf{Fecha de entrega:} 13 de diciembre, 23:59 hrs. Vía Aula.
    
    \item Revise 2 veces que el proyecto (zip) que vaya a entregar sea correcto y que contenga todas sus respuestas.
    
    \item Se descontarán 10 puntos por atraso desde las 00:01 hrs hasta las 01:00 hrs, y se irán restando sucesivamente 10 puntos de su nota por cada hora de atraso.
\end{itemize}
    


\end{document}
