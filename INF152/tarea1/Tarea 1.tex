\documentclass[letterpaper,10pt]{article}
\usepackage[utf8]{inputenc}
\usepackage[spanish,mexico]{babel}
\usepackage{amsmath}
\usepackage{amsfonts}
\usepackage{enumerate}
\usepackage{float}
\usepackage{indentfirst}
\usepackage{graphicx}
\usepackage{url}
\usepackage{multicol}
\usepackage{geometry}
\usepackage{fullpage}
\usepackage{hyperref}
\usepackage{tikz}
%\usetikzlibrary{shapes}
\usetikzlibrary{arrows,automata}

\tikzset{
    %Define standard arrow tip
    >=stealth',
    % Define arrow style
    pil/.style={
           ->,
           thick,}
}

\def\checkmark{\tikz\fill[scale=0.4](0,.35) -- (.25,0) -- (1,.7) -- (.25,.15) -- cycle;}


\begin{document}
    
\thispagestyle{empty}
 	
\begin{minipage}[t]{0.6\textwidth}

{\LARGE \textbf{INF152} Estructuras Discretas}

{\large Profesores: Margarita Bugueño, Sebastián Gallardo.}\\
{\large Ayudantes: Valentina Aróstica, Bryan González, Sofía Mañana y Sofía Riquelme}

Universidad T\'ecnica Federico Santa Mar\'{\i}a

Departamento de Inform\'atica -- CSJ - CC 

Septiembre 24, 2021

\end{minipage}
\hfill
\begin{minipage}[t]{0.3\textwidth}
Nombre:

\begin{tabular}{|c|}\hline
    
Diego Eduardo Paz Letelier\\\hline
\end{tabular}

\vspace{0.1cm}

Rol:

\begin{tabular}{|c|c|c|c|c|c|c|c|c|c|c|}\hline

2 & 0 & 2 & 0 & 0 & 4 & 5 & 0 & 2 & - & K\\\hline
\end{tabular}
\end{minipage}

\vspace{0.3cm}

\begin{center}
    \huge Tarea 1
\end{center}

\counterwithin*{equation}{section}
\counterwithin*{equation}{subsection}

\section{Enunciado 1}

\subsection{Formalizaci\'on}
\begin{minipage}[t]{0.4\textwidth}
\begin{itemize}
    \item \textit{P}(\textit{x}): \textit{x} dice la verdad.
    \item a: Bryan.
    \item b: Valentina.
    \item c: Sof\'ia.
    \item d: Claudio.
\end{itemize}
\end{minipage}
\begin{minipage}[t]{0.5\textwidth}
Formalizando los enunciados obtenemos que:
\begin{equation}
    P(b) \Rightarrow P(a)
\end{equation}
\begin{equation}
    P(a) \Leftrightarrow \neg P(c)
\end{equation}
\begin{equation}
    P(c) \wedge  P(d)
\end{equation}
\begin{equation}
    P(d) \Rightarrow \neg P(a)
\end{equation}
\end{minipage}

\subsection{Demostraci\'on}
\begin{minipage}[c]{0.4\textwidth}
\begin{equation*}
    P(c) \wedge  P(d)
\end{equation*}
\begin{equation*}
    P(d)
\end{equation*}
\begin{equation*}
    \neg P(a)
\end{equation*}
\begin{equation*}
    \neg P(b)
\end{equation*}
\begin{equation*}
    \neg P(a) \wedge \neg P(b)
\end{equation*}
\end{minipage}
\begin{minipage}[c]{0.5\textwidth}
\begin{enumerate}[I{.-}]
    \item Hip\'otesis
    \item Simplificaci\'on de (3) en base a I.-
    \item Modus Ponens de (4) y II.-
    \item Modus Tollens de (1) y III.-
    \item Ley de combinaci\'on de III.- y IV.-
\end{enumerate}
\end{minipage}
\vspace{0.3cm}
\newline
Podemos concluir entonces que ni Bryan ni Valentina dicen la verdad.


\section{Enunciado 2}

\subsection{a)}

\subsubsection{Formalizaci\'on}

\subsubsection{Demostraci\'on}

\subsection{b)}

\subsubsection{Formalizaci\'on}

\subsubsection{Demostraci\'on}

\end{document}